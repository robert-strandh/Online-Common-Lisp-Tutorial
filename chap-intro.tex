\chapter{Introduction}

\section{How to accomplish online teaching}

In the September 2015, Philip Compeau and Pavel A. Pevzener wrote an
article entitled ``Life after MOOCs'' \cite{10.1145/2686871}.  In that
article, they not only criticize MOOCs, i.e. ``Massive Open Online
Courses'', but they have a proposal of how to accomplish good online
teaching.  We largely agree with them.

In their proposal, they realize that it is not enough for a team of
teachers to make their teaching material available online, nor is it
enough to record traditional lectures and make them available as
movies on some website.  Instead, they suggest that a team of experts
must be in charge in order to produce high-quality online teaching.
Such a team would have to consist of the following experts:

\begin{itemize}
\item Domain experts, i.e., experts in the field to be taught.
\item Pedagogues, i.e. people trained in how material is best
  presented, and in what order, as well as how and when to test
  acquired knowledge.
\item Experts in web design.
\item Engineers to accomplish the physical realization of the
  presentation such as web pages, figures, animation snippets, video
  snippets, etc.
\end{itemize}

The authors of the article estimates the cost of a correctly made
online course to be roughly 1 million USD.  For that reason, it is out
of scope for must individual universities to produce such a thing.  It
would require collaboration between several universities to share the
cost.

\section{Online tutorial for \commonlisp{}}

While it is clear that it would be very hard to find 1 million USD in
order to realize an online course for teaching \commonlisp{}, we think
that the ideas of Compeau and Pevzner can be used to accomplish
something in the right spirit, but much less ambitious, to serve as an
online tutorial for \commonlisp{}.

People using \commonlisp{}, collectively possess all the expertise
required in the proposal of Compeau and Pevzner, and there would be no
requirement to create an entire site by a certain date.  Instead, such
a site could evolve over time, thereby limiting the individual burden
of each participant.

Furthermore, in this document, we suggest creating such a site in the
form of a large collection of \emph{nodes} of limited size.  Each node
would represent only a few minutes worth of information, again making
the production of a single node a more reasonable task.  The nodes
would not be directly linked.  Instead, separate navigation plans
would be provided, and several different such plans could be created,
according to the objectives of each creator.
