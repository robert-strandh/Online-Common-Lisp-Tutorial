\chapter{Recursive version of \texttt{length}}

This node is meant to illustrate recursion \emph{operationally} using
the \texttt{length} function as an example.

The code for the \texttt{length} function is:

\begin{verbatim}
  (defun length (list)
    (if (null list)
        0
        (length (rest list))))
\end{verbatim}

We propose to illustrate a call like this:

\begin{verbatim}
  (length *l*)
\end{verbatim}

Where \texttt{*l*} is a proper list containing $3$ elements.

\begin{figure}
\begin{center}
\inputfig{fig-recursion-length-1.pdf_t}
\end{center}
\caption{\label{fig-recursion-length-1}
Recursive length 1}
\end{figure}

\begin{figure}
\begin{center}
\inputfig{fig-recursion-length-2.pdf_t}
\end{center}
\caption{\label{fig-recursion-length-2}
Recursive length 2}
\end{figure}

\begin{figure}
\begin{center}
\inputfig{fig-recursion-length-3.pdf_t}
\end{center}
\caption{\label{fig-recursion-length-3}
Recursive length 3}
\end{figure}

\begin{figure}
\begin{center}
\inputfig{fig-recursion-length-4.pdf_t}
\end{center}
\caption{\label{fig-recursion-length-4}
Recursive length 4}
\end{figure}

\begin{figure}
\begin{center}
\inputfig{fig-recursion-length-5.pdf_t}
\end{center}
\caption{\label{fig-recursion-length-5}
  Recursive length 5}
\end{figure}


\begin{figure}
\begin{center}
\inputfig{fig-recursion-length-6.pdf_t}
\end{center}
\caption{\label{fig-recursion-length-6}
  Recursive length 6}
\end{figure}


\begin{figure}
\begin{center}
\inputfig{fig-recursion-length-7.pdf_t}
\end{center}
\caption{\label{fig-recursion-length-7}
  Recursive length 7}
\end{figure}


\begin{figure}
\begin{center}
\inputfig{fig-recursion-length-8.pdf_t}
\end{center}
\caption{\label{fig-recursion-length-8}
  Recursive length 8}
\end{figure}


\begin{figure}
\begin{center}
\inputfig{fig-recursion-length-9.pdf_t}
\end{center}
\caption{\label{fig-recursion-length-9}
  Recursive length 9}
\end{figure}

