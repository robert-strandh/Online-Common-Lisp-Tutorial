\documentclass{beamer}
\usepackage[utf8]{inputenc}
\beamertemplateshadingbackground{red!10}{blue!10}
%\usepackage{fancybox}
\usepackage{epsfig}
\usepackage{verbatim}
\usepackage{url}
%\usepackage{graphics}
%\usepackage{xcolor}
\usepackage{fancybox}
\usepackage{moreverb}
%\usepackage[all]{xy}
\usepackage{listings}
\usepackage{filecontents}
\usepackage{graphicx}

\lstset{
  language=Lisp,
  basicstyle=\scriptsize\ttfamily,
  keywordstyle={},
  commentstyle={},
  stringstyle={}}

\def\inputfig#1{\input #1}
\def\inputeps#1{\includegraphics{#1}}
\def\inputtex#1{\input #1}

\inputtex{logos.tex}

%\definecolor{ORANGE}{named}{Orange}

\definecolor{GREEN}{rgb}{0,0.8,0}
\definecolor{YELLOW}{rgb}{1,1,0}
\definecolor{ORANGE}{rgb}{1,0.647,0}
\definecolor{PURPLE}{rgb}{0.627,0.126,0.941}
\definecolor{PURPLE}{named}{purple}
\definecolor{PINK}{rgb}{1,0.412,0.706}
\definecolor{WHEAT}{rgb}{1,0.8,0.6}
\definecolor{BLUE}{rgb}{0,0,1}
\definecolor{GRAY}{named}{gray}
\definecolor{CYAN}{named}{cyan}

\newcommand{\orchid}[1]{\textcolor{Orchid}{#1}}
\newcommand{\defun}[1]{\orchid{#1}}

\newcommand{\BROWN}[1]{\textcolor{BROWN}{#1}}
\newcommand{\RED}[1]{\textcolor{red}{#1}}
\newcommand{\YELLOW}[1]{\textcolor{YELLOW}{#1}}
\newcommand{\PINK}[1]{\textcolor{PINK}{#1}}
\newcommand{\WHEAT}[1]{\textcolor{wheat}{#1}}
\newcommand{\GREEN}[1]{\textcolor{GREEN}{#1}}
\newcommand{\PURPLE}[1]{\textcolor{PURPLE}{#1}}
\newcommand{\BLACK}[1]{\textcolor{black}{#1}}
\newcommand{\WHITE}[1]{\textcolor{WHITE}{#1}}
\newcommand{\MAGENTA}[1]{\textcolor{MAGENTA}{#1}}
\newcommand{\ORANGE}[1]{\textcolor{ORANGE}{#1}}
\newcommand{\BLUE}[1]{\textcolor{BLUE}{#1}}
\newcommand{\GRAY}[1]{\textcolor{gray}{#1}}
\newcommand{\CYAN}[1]{\textcolor{cyan }{#1}}

\newcommand{\reference}[2]{\textcolor{PINK}{[#1~#2]}}
%\newcommand{\vect}[1]{\stackrel{\rightarrow}{#1}}

% Use some nice templates
\beamertemplatetransparentcovereddynamic

\newcommand{\A}{{\mathbb A}}
\newcommand{\degr}{\mathrm{deg}}

\title{Mathematical induction}

%\inputtex{macros.tex}


\begin{document}
\frame{
\titlepage
}

\setbeamertemplate{footline}{
\vspace{-1em}
\hspace*{1ex}{~} \GRAY{\insertframenumber/\inserttotalframenumber}
}

\frame{
\frametitle{Example}

Let us say that we would like to prove the proposition that a
\emph{perfect binary tree} of hight $h$ has $n = 2^{h+1} - 1$ nodes.

\vskip 1cm

A \emph{perfect binary tree} is a binary tree in which all interior
nodes have two children and all leaves have the same depth
[Wikipedia].
}

\frame{
\frametitle{Perfect binary tree}
\begin{figure}
\begin{center}
\inputfig{fig-perfect-binary-tree.pdf_t}
\end{center}
\end{figure}
}

\frame{
\frametitle{First check some examples}
Start by convincing yourself that the proposition might be true with
some examples.
}

\frame{
\frametitle{Tree with height 0}
\begin{figure}
\begin{center}
\inputfig{fig-tree0.pdf_t}
\end{center}
\end{figure}
$2^{h+1} - 1 = 2^{0+1} - 1 = 2^{1} - 1 = 2 - 1 = 1 = n$
}

\frame{
\frametitle{Tree with height 1}
\begin{figure}
\begin{center}
\inputfig{fig-tree1.pdf_t}
\end{center}
\end{figure}
$2^{h+1} - 1 = 2^{1+1} - 1 = 2^{2} - 1 = 4 - 1 = 3 = n$
}

\frame{
\frametitle{Tree with height 2}
\begin{figure}
\begin{center}
\inputfig{fig-tree2.pdf_t}
\end{center}
\end{figure}
$2^{h+1} - 1 = 2^{2+1} - 1 = 2^{3} - 1 = 8 - 1 = 7 = n$
}

\frame{
\frametitle{Tree with height 3}
\begin{figure}
\begin{center}
\inputfig{fig-tree3.pdf_t}
\end{center}
\end{figure}
$2^{h+1} - 1 = 2^{3+1} - 1 = 2^{4} - 1 = 16 - 1 = 15 = n$
}

\frame{
\frametitle{Mathematical induction}
\begin{enumerate}
\item Establish a \emph{base case}.
\item Establish an \emph{induction hypothesis}.
\item Using the induction hypothesis, prove that the proposition is
  true.
\end{enumerate}
}

\frame{
\frametitle{Mathematical induction (base case)}
We check that the proposition is true for the smallest possible value of
$h$, which is $h = 0$.
\vskip 1cm
But our first figure shows that for $h = 0$, $n = 1$, so the proposition
is true for the base case.
}

\frame{
\frametitle{Mathematical induction (induction hypothesis)}
Assume that the proposition is true for $h = 0, 1, \cdots, k$.
}

\frame{
\frametitle{Mathematical induction (proof)}
Prove that the proposition is also true for $h = k+1$.
\vskip 0.5cm
So, suppose we have a tree with height $h = j = k + 1$, where $k \ge 0$.
Then we have the following situation:
\begin{figure}
\begin{center}
\inputfig{fig-induction-hypothesis.pdf_t}
\end{center}
\end{figure}
By the induction hypothesis, each child of the root has $n = 2^{k+1} -
1$ nodes.  The total number of nodes in this tree is $2(2^{k+1} - 1) +
1 = 2^{(k+1+1)} - 2 + 1 = 2^{j+1} - 1$ which concludes the proof.
}

\frame{
  \frametitle{Mathematical induction (summary)}
  \begin{itemize}
  \item We show that the proposition is true for some \emph{base
    case}.  When the domain is the non-negative integers, the base
    case is often $0$ as in this case for $h$.
  \item We establish an \emph{induction hypothesis} that the
    proposition is true for values (in this case $h$) of $0, 1,
    \cdots, k$.
  \item Given the induction hypothesis, we prove that the proposition
    is also true for $k+1$.
  \item The previous steps taken together, we have shown that the
    proposition is true, first for $h=0$.  Then using the induction
    step, we know that it is true for $h=1$, etc.  We conclude that
    the proposition is true for all non-negative integers.
  \end{itemize}
  }

\end{document}
